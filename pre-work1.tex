\documentclass{article}
\title{Technology Research Preparation \\
Pre-Work 1: Setting up for Assignment 2 }
\author{Barry Jay}
\date{Autumn 2016}

\begin{document}
\maketitle

\subsection*{The Skeleton of your Literature-Based Report}

Your task is to produce a {\em skeleton} of the literature-based
report that you will produce for Assignment 2. That is, it will have
all of the structure of the report (the bones) but almost no content
(no flesh).  The goal here is to produce a report of professional
standard.  There are quite a few small tasks to complete. By
completing them now, you can focus on the content of your report
later.  The main things you have to do are:
\begin{enumerate}
\item choose a word-processor, e.g. Latex or Word
\item
choose a style file or template for your report
\item pick a topic for your report (this may change)
\item
fill in your name, the report title, and the submission date (check the subject outline)
\item
write a paragraph explaining why you are interested in this topic. 
\item
pick one scholarly paper on this topic. 
\item
make a citation of this paper in a second paragraph.
\item create the bibliography for the report.
\item 
check that everything has compiled correctly. 
\end{enumerate}


\subsection*{An example in latex}

The example below is done in latex. You are welcome to use Word if you
prefer, in which case you probably want to use EndNote for your references. 
If you want to try out latex, you will need to make sure that it is installed on your machine. 
The document you are reading, pre-work1.pdf, was created from
pre-work1.tex (available on UTSOnline) by executing the following commands
\begin{verbatim}
pdflatex pre-work1
bibtex pre-work1
pdflatex pre-work1
pdflatex pre-work1
\end{verbatim}
The first use of ``pdflatex'' does a first pass through the document.
Most of the formatting is done on this pass, including production of a
pdf file, but section numbers, reference numbers etc are represented
by question marks, whose values will be determined in subsequent
passes. Then ``bibtex pre-work1'' creates the bibliography from a
specified bibliographic database, here called two\_papers.bib. Two more passes get all the
numbering and references settled.
Now let us go through the steps.

\begin{enumerate}
\item I create a file
  pre-work1.tex, also available on UTS online.
\item
I an using the article style, as indicated by the opening line
\begin{verbatim}
\documentclass{article}
\end{verbatim}
\item
My topic is pre-work1 for TRP. 
\item 
The title, author and date are given by 
\begin{verbatim}
\title{Technology Research Preparation \\
Pre-Work 1: Setting up for  Assignment 2 }
\author{Barry Jay}
\date{Autumn 2016}
\end{verbatim}
\item The paragraphs you see are mainly straight typing but the {\tt
    enumerate} environemnt is used to make a numbered list, and {\tt
    verbatim} is used for quotation. 
\item My scholarly paper is not actually about TRP. I wrote a book
  about the foundations of computing, about {\em pattern calculus}
  \cite{pcb} which will do to illustrate. 
\item
In pre-work1.tex, the citation is given by 
\begin{verbatim}
... about {\em pattern calculus} \cite{pcb}  ...
\end{verbatim}
All $\backslash$ are used to indicate latex commands. The ``em'' is
used for emphasis. The ``cite'' is used to indicate a citation, whose
key is ``pcb'' (for pattern calculus book).
\item
The bibliography is set up using 
\begin{verbatim}
\bibliographystyle{plain}
\bibliography{two_papers}
\end{verbatim}
The first line sets the bibliographic style. The second links to the
bibliographic data in ``two\_papers.bib''. You will need to create
such a database for your papers.  That doesn't mean that you should
type the data by hand. Often, the publisher will do it for you. In
this case Springer supplies the information but you need to do a lot
of cut and paste. So here is another paper I wrote \cite{JK09} whose
publiser provides a bibtex citation that you can just drop into your
bib file. 
\end{enumerate}

If this is your first time using latex then take a copy of pre-work1.tex and check that you can make it work for you. 



\bibliographystyle{plain}
\bibliography{two_papers}


\end{document}

choose a style file or template for your report
\item pick a topic for your report 
\item
fill in your name, the report title, and the submission date (check the subject outline)



 is an example of the

Here is an example.



process the citation using a bibliographic tool, such as bibtex or EndNote. 
A theory of computation based on quantum logic (I)
\item
 supervisor's name, intended submission date; put your current research proposal in Chapter 1; put your current literature review (e.g. from TRP) from TRP in Chapter 2create six chapters, including Introduction and Literature Review